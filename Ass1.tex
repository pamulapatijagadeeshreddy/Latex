\documentclass[12pt,-letter paper]{article}
\usepackage{siunitx}
\usepackage{setspace}
\usepackage{gensymb}
\usepackage{xcolor}
\usepackage{caption}
%\usepackage{subcaption}
\doublespacing
\singlespacing
\usepackage[none]{hyphenat}
\usepackage{amssymb}
\usepackage{relsize}
\usepackage[cmex10]{amsmath}
\usepackage{mathtools}
\usepackage{amsmath}
\usepackage{commath}
\usepackage{amsthm}
\interdisplaylinepenalty=2500
%\savesymbol{iint}
\usepackage{txfonts}
%\restoresymbol{TXF}{iint}
\usepackage{wasysym}
\usepackage{amsthm}
\usepackage{mathrsfs}
\usepackage{txfonts}
\let\vec\mathbf{}
\usepackage{stfloats}
\usepackage{float}
\usepackage{cite}
\usepackage{cases}
\usepackage{subfig}
%\usepackage{xtab}
\usepackage{longtable}
\usepackage{multirow}
%\usepackage{algorithm}
\usepackage{amssymb}
%\usepackage{algpseudocode}
\usepackage{enumitem}
\usepackage{mathtools}
%\usepackage{eenrc}
%\usepackage[framemethod=tikz]{mdframed}
\usepackage{listings}
%\usepackage{listings}
\usepackage[latin1]{inputenc}
%%\usepackage{color}{   
%%\usepackage{lscape}
\usepackage{textcomp}
\usepackage{titling}
\usepackage{hyperref}
%\usepackage{fulbigskip}   
\usepackage{tikz}
\usepackage{graphicx}
\lstset{
  frame=single,
  breaklines=true
}
\let\vec\mathbf{}
\usepackage{enumitem}
\usepackage{graphicx}
\usepackage{siunitx}
\let\vec\mathbf{}
\usepackage{enumitem}
\usepackage{graphicx}
\usepackage{enumitem}
\usepackage{tfrupee}
\usepackage{amsmath}
\usepackage{amssymb}
\usepackage{mwe} % for blindtext and example-image-a in example
\usepackage{wrapfig}
\graphicspath{{figs/}}
\providecommand{\mydet}[1]{\ensuremath{\begin{vmatrix}#1\end{\vmatrix}}}
\providecommand{\myvec}[1]{\ensuremath{\begin{bmatrix}#1\end{\bmatrix}}}
\providecommand{\cbrak}[1]{\ensuremath{\left\{#1\right\}}}
\providecommand{\brak}[1]{\ensuremath{\left(#1\right)}}
\begin{document}
\begin{enumerate}
\item \textbf{Assertion (A):} If the graph of a polynomial touches x-axisat only one point, then the polynomial cannot be a quadratic polynomial.

	\textbf{Reason (R):} A polynomial of degree $n\brak{n > 1}$ can have at most n zeroes.
\item solve the following system of linear equcations $7x-2y=5$ and verify your answer.
\item At in a pack of 52 playing cards one card is lost. From the remain cards, a card is drawn at random.
Find the probability that the drawn card is queen of heart, if the lost card is a black card.
\item\textbf{ Evaluate:} $2\sqrt{2} \cos 45\degree \sin 10\degree + 2\sqrt{3} \cos 30\degree$

\item If  $A = 60\degree$ and  $B = 30\degree$, verify that : $\sin\brak{A + B} = \sin A \cos B + \cos A \sin B$

\item In  the given figure, $ABCD$ is a qadrilateral.diagonal $BD$ bisects $\angle B$ and $\angle D$ both. 
\begin{figure}[!ht]
\centering
\includegraphics[width=\columnwidth]{image 1.jpg}
\label{fig:image 1}
\caption{image 1}
\end{figure}\\
\text Prove that :\\	
\begin{enumerate}	
\item$\triangle ABD \sim \triangle CBD $\\
\item $AB = BC$
\end{enumerate}
\newpage
\item Prove that $ 5 - 2\sqrt{3}$ is an irrational number. It is given that $\sqrt{3}$  is an irrational number.

\item show that the number $5x11x17+3x11$ is a composite number.
\item Find the ratio in which the point $\brak{8, y}$ divides the line segment joining the points $\brak{1, 2}$ and $\brak{2, 3}$.. Also, find the value of  $y$.
\newpage
\item $ABCD$ is a rectangle formed by the points $ A\brak{-1,-1}, B\brak{-1, 6}, C\brak{3, 6}$ and $D\brak{3, 1}, P, Q, R$ and $S$ are mid-points of sides $AB, BC, CD$ and $DA$ respectively.Show that diagonals of the quadrilateral $PQRS$ bisect each other.
\item In a teachers workshop,the number of teachers teaching French,Hindi and English are $48$, $80$ and $144$ respectively. Find the minimum number of root required if in each room the same number of teachers are seated and all of them are of the same subject.
\item Prove that : $\frac{\tan{\theta}}{1 - \cot{\theta}} + \frac{\cot{\theta}}{1 - \tan{\theta}} = 1 + \sec{\theta}\csc{\theta}$
\item Three years ago, Rashmi was thrice as old as Nazma. Ten years later, Rashmi will be twice as old as Nazma. How old are Rashmi and Nazma now?
\item In the given figure,$AB$ is a diameter of the circle $O. AQ, BP$ and $PQ$ are tangents to the circle.prove that $\angle POQ = 90\degree$.
\begin{figure}[!ht]
\centering
\includegraphics[width=\columnwidth]{Image 2.jpg}
\label{fig:Image 2}
\caption{image 2}
\end{figure}
\item  A circle with centre $O$ and radius $8\mathrm{cm}$ is inscribed in a quadrilateral $ABCD$ in which $P, Q, R, S$ are the points of contact as shown. If $AD$ is perpendicular to $DC, BC = 30\mathrm{cm}$ and $BS = 24\mathrm{cm}$, then find the length $DC$.
\begin{figure}[!ht]
\centering
\includegraphics[width=\columnwidth]{Image 3.jpg}
\label{fig:Image 3}
\caption{image 3}
\end{figure}
\item The difference between the outer and inner radii of a hollow right circular cylinder of length $14\mathrm{cm}$ is $1\mathrm{cm}$. If the volume of the metal used in making the cylinder is $176\mathrm{cm}^3$, find the outer and inner radii of the cylinder.
\newline
\item An arc of a circle of radius $21\mathrm{cm}$ subtends an angle of $60\degree$ at the centre. Find:
\begin{enumerate}
\item the length of the arc.
\item the area of the minor segment of the circle made by the corresponding chord.
\end{enumerate}
\item  The sum of first and eights of an A.P. is $32$ and their product is $60$. Find the first term  and common difference of the A.P. Hence,also find the sum of its first $20$ terms.
\item In an A.P. of $40$ terms, the sum of first $9$ terms is $153$ and the sum of last $6$ terms is $687$. Determine the first term and common difference of A.P. Also, find the sum of all the terms of the A.P.
\item If a line is drawn parallel to one side of a triangle to intersect the other two sides in distinct points, then prove that the other two sides are divided in the same ratio.
\item In the given figure $PA,QB$ and $RC$ are each perpenicular to $AC$. if $AP = x,bq=y and cr =z$,then prove that $\frac{1}{x} + \frac{1}{z} = \frac{1}{y}$                                               
\begin{figure}[!ht]
\centering
\includegraphics[width=\columnwidth]{image 4.jpg}
\label{fig:image 4}
\caption{image 4}
\end{figure}
\newpage
\item A pole $6m$ high is fixed on the top of a tower. The angle of elevation of the top of the pole observed from a point $P$ on the ground is $60\degree$ and the angle of depression of the point $P$ from the top of the tower is $45\degree$. Find the height of the tower and the distance of point $P$ from the foot of the tower.
\newpage
\item A rectangular floor area can be completely tiled with $200square$ tiles. If the side length of each tile is increased by $1$ unit, it would take only $128$ tiles to cover the floor.
\begin{figure}[!ht]
\centering
\includegraphics[width=\columnwidth]{image 5.jpg}
\label{fig:image5}
\caption{image5} 
\end{figure}
\begin{enumerate}
\item Assuming the original length of each side of a tile be x units, make a quadratic equation from the above information.\\
\item Write the corresponding quadratic equation in standard form.\\
\item Find the value of x, the length of side of a tile by factorisation.
\end{enumerate}
\item solve the quadratic equcation for x, using quardratic formula.
\newpage
\item BINGO is game of chance. The host has $75$ balls numbered $1$ through $75$. Each player has a BINGO card with some numbers written on it.
The participant cancels the number on the card when called out a number written on the ball selected at random. Whosoever cancels all the numbers on his/her card, says BINGO and wins the game.

\begin{figure}[!ht]
\centering
\includegraphics[width=\columnwidth]{image 6.jpg}
\label{fig:image 6}
\caption{image 6}
\end{figure}

\text The table given below,shows the data of one game where $48$ ball were used before Tara said "BINGO".
\begin{center}
\begin{tabular}{|c|c|}
\hline
Numbers announced & Number of times \\
\hline
0-15 & 8 \\
\hline
15-30 & 9 \\
\hline
30-45 & 9 \\
\hline
45-60 & 10 \\
\hline
60-75 & 12 \\
\hline
\end{tabular}
\end{center}
 Based on the above information, answer the following:\\
\begin{enumerate}
\item Write the median class.\\
\item When first ball was picked up,what was the probability of calling out an even umber?\\
\item Find median and mode of an the given data.
\end{enumerate}

\item A backyard is in the shape of a triangle $ABC$ with right angle at $B AB = 7 m$ and $BC = 15 m$. A circular pit was dug inside it such that it touches the walls $AC. BC$ and $AB$ at $P, Q$ and $R$ respectively such that $AP=xm$.
\begin{figure}[!ht]
\centering
\includegraphics[width=\columnwidth]{image 7.jpg}
\label{fig:image 7}
\caption{image 7}
\end{figure}
\text Based on the above information,answer the following question:\\
 \begin{enumerate}
\item find the length of AR in terms of $x$.\\
\item write the type of quardrilateral of $BQOR$.\\
\item Find the length C in terms of x and hence find the value of $x$.\\
\item Find $x$ and hence find the radius r of circle.
\end{enumerate}
\end{enumerate}
\end{document}
