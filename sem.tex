\documentclass{article}
\usepackage{gensymb}
\usepackage{amsmath}
\usepackage{amsfonts}
\usepackage{graphicx}
\begin{document}
\begin{enumerate}
\item \textbf{Assertion (A):} If the graph of a polynomial touches x-axis at only one point, then the polynomial cannot be a quadratic polynomial.

\textbf{Reason (R):} A polynomial of degree $n(n >1)$ can have at most n zeroes.
\item solve the following system of linear equcations $7x-2y=5$ and verify your answer.
\item\text{At in a pack of 52 playing cards one card is lost. From the remain cards, a card is drawn at random.}

\text{Find the probability that the drawn card is queen of heart, if the lost card is a black card.}
\item\textbf{ Evaluate:} $2\sqrt{2} cos 45\degree sin 10\degree + 2\sqrt{3} cos 30\degree$

\item If  $A = 60\degree and  B = 30\degree$, verify that : $sin(A + B) = sin A cos B + cos A sin B$
\newpage
\item In  the given figure, ABCD is a qadrilateral.diagonal BD bisects $\angle B$ and $\angle D$ both. 
\begin{figure}[h!]
\centering
\includegraphics[width=\columnwidth]{image 1.jpg}
\label{fig:image 1}
\caption{image 1}
\end{figure}\\
\text Prove that :\\	
		(i)$\triangle ABD \sim \triangle CBD $\\
		(ii)AB = BC
\item Prove that $ 5 - 2\sqrt{3}$ is an irrational number. It is given that $\sqrt{3}$  is an irrational number.

\item show that the number $5x11x17+3x11$ is a composite number.
\item Find the ratio in which the point $(8, y)$ divides the line segment joining the points $(1, 2)$ and $(2, 3)$.. Also, find the value of  $y$.

\item ABCD is a rectangle formed by the points A (-1,-1), B (-1, 6), C (3, 6) and D (3, 1), P, Q, R and S are mid-points of sides AB, BC, CD and DA respectively.Show that diagonals of the quadrilateral PQRS bisect each other.
\item\text{} In a teachers workshop,the number of teachers teaching French,Hindi and English are 48, 80 and 144 respectively. Find the minimum number of rooms required if in each room the same number of teachers are seated and all of them are of the same subject.
\item Prove that : $\frac{tan{\theta}}{1-cot{\theta}} + \frac{cot{\theta}}{1-tan{\theta}} = 1 + sec\theta cosec\theta$
\item Three years ago, Rashmi was thrice as old as Nazma. Ten years later, Rashmi will be twice as old as Nazma. How old are Rashmi and Nazma now?
\item In the given figure,AB is a diameter of the circle O. AQ, BP and PQ are tangents to the circle.prove that $\angle POQ = 90\degree$.
\begin{figure}[h!]
\centering
\includegraphics[width=\columnwidth]{Image 2.jpg}
\label{fig:Image 2}
\caption{image 2}
\end{figure}
\newpage
\item  A circle with centre O and radius 8 cm is inscribed in a quadrilateral ABCD in which P, Q, R, S are the points of contact as shown. If AD is perpendicular to DC, BC = 30 cm and BS = 24 cm, then find the length DC.
\begin{figure}[h!]
\centering
\includegraphics[width=\columnwidth]{Image 3.jpg}
\label{fig:Image 3}
\caption{image 3}
\end{figure}
\item The difference between the outer and inner radii of a hollow right circular cylinder of length 14 cm is 1 cm. If the volume of the metal used in making the cylinder is 176 cm³, find the outer and inner radii of the cylinder.
\newpage
\item An arc of a circle of radius 21 cm subtends an angle of 60\textdegree at the centre. Find:\\
\text{(i)} the length of the arc.\\
\text{(ii)} the area of the minor segment of the circle made by the corresponding chord. 
\item  The sum ofv first and eights of an A.P. is 32 and their product is 60. Find the first term  and common difference of the A.P. Hence,also find the sum of its first 20 terms.
\item In an A.P. of 40 terms, the sum of first 9 terms is 153 and the sum of last 6 terms is 687. Determine the first term and common difference of A.P. Also, find the sum of all the terms of the A.P.
\item If a line is drawn parallel to one side of a triangle to intersect the other two sides in distinct points, then prove that the other two sides are divided in the same ratio.
\item In the given figure PA,QB and RC are ach perpenicular to AC. if AP = x,bq=y and cr =z,then prove that $\frac{1}{x} + \frac{1}{z} = \frac{1}{y}$                                               
\begin{figure}[h!]
\centering
\includegraphics[width=\columnwidth]{image 4.jpg}
\label{fig:image 4}
\caption{image 4}
\end{figure}
\newpage
\item A pole 6m high is fixed on the top of a tower. The angle of elevation of the top of the pole observed from a point P on the ground is 60° and the angle of depression of the point P from the top of the tower is 45°. Find the height of the tower and the distance of point P from the foot of the tower.
\item A rectangular floor area can be completely tiled with 200 square tiles. If the side length of each tile is increased by 1 unit, it would take only 128 tiles to cover the floor.

\begin{figure}[h!]
\centering
\includegraphics[width=\columnwidth]{image 5.jpg}
\label{fig:image5}
	\caption{image5} 
\end{figure}
(i) Assuming the original length of each side of a tile be x units, make a quadratic equation from the above information.\\
(ii) Write the corresponding quadratic equation in standard form.\\
\text{(iii)} Find the value of x, the length of side of a tile by factorisation.
\item solve the quadratic equcation for x, using quardratic formula.
\newpage
\item BINGO is game of chance. The host has 75 balls numbered 1 through 75. Each player has a BINGO card with some numbers written on it.
The participant cancels the number on the card when called out a number written on the ball selected at random. Whosoever cancels all the numbers on his/her card, says BINGO and wins the game.

\begin{figure}[h!]
\centering
\includegraphics[width=\columnwidth]{image 6.jpg}
\label{fig:image 6}
\caption{image 6}
\end{figure}

\text The table given below,shows the data of one game where 48 ball were used before Tara said "BINGO".
\begin{center}
\begin{tabular}{|c|c|}
\hline
Numbers announced & Number of times \\
\hline
0-15 & 8 \\
\hline
15-30 & 9 \\
\hline
30-45 & 9 \\
\hline
45-60 & 10 \\
\hline
60-75 & 12 \\
\hline
\end{tabular}
\end{center}
\text Based on the above information, answer the following:\\
(i)  Write the median class.\\
(ii) When first ball was picked up,what was the probability of calling out an even umber?\\
(iii) Find median and mode of an the given data.
\newpage
\item A backyard is in the shape of a triangle ABC with right angle at B AB = 7 m and BC = 15 m. A circular pit was dug inside it such that it touches the walls AC. BC and AB at P, Q and R respectively such that AP=xm.
\begin{figure}[h!]
\centering
\includegraphics[width=\columnwidth]{image 7.jpg}
\label{fig:image 7}
\caption{image 7}
\end{figure}
\text Based on the above information,answer the following question:\\
(i) find the length of AR in terms of x.\\
(ii) write the type of quardrilateral of bQor.\\
(iii) Find the length C in terms of x and hence find the value of x.\\
(iv) Find x and hence find the radius r of circle.
\end{enumerate}
\end{document}
